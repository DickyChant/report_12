\documentclass{ctexart}
    \usepackage{mathrsfs}
    \usepackage{multirow}
    \usepackage{graphicx}
    \usepackage{array}
    \usepackage{makecell}
    \usepackage{amsmath}
    \usepackage{booktabs}
    \usepackage{float}
    \usepackage{diagbox}
    \newcommand\mgape[1]{\gape{$\vcenter{\hbox{#1}}$}}
    \newcommand\Ronum[1]{\uppercase\expandafter{\romannumeral #1\relax}}
    \newcommand\ronum[1]{\romannumeral #1\relax}
    \author{钱思天\ 1600011388 No.8}
    \title{实验九\ 刚体转动实验 \ 实验报告}
    \begin{document}
      \maketitle
      \section{实验数据与处理}
      \subsection{测量数据列表}
      \subsubsection{实验内容(2)}
      \subsubsection{实验内容(3)}
      \subsubsection{实验内容(4)}
      \subsection{实验数据处理}
      \subsubsection{验证线性关系}
      \subsubsection{线性拟合求转动惯量}
      \section{分析与讨论}
      \subsection{减少误差}
      \paragraph{系统误差}为减少系统误差,应尽量满足得出实验结论所需的一系列近似条件;同时,应使$OO_1$轴尽量竖直,绳子尽量水平以及让绳子密绕
      在塔轮上等。
      \paragraph{随机误差}掐秒表时应集中注意力,找准落地点等。
      \subsection{思考题(5)}
      实验(3)中,考虑塔轮半径的改变,其摩擦力矩$M_{\mu}$改变且越来越大,因此有:
      $$(mg-f)r-M_{others}=I\dot\omega=\frac{2hI}{rt^2}$$
      $$\Rightarrow I_2=\frac{mgk}{2h}=\frac{mg}{mg-f}I>I$$
      同时,实验(2)中,塔轮半径不变,因此摩擦力矩$M_{\mu}$可认为不变,故可认为$I_1=I$,
      则有$I_2>I_1$。
      \section{收获与感想}
      在预习这个实验的时候,我情不自禁地想起来高一时所做的,验证牛顿第二定律的实验,也是用重物的重力做外力并计算加速度。

      在我看来,这两个实验有很多相似的地方,譬如都要使加速度远小于重力加速度等。

      其实从实验研究的对象,也能感受到高中与大学所学习内容的区别,从可当作质点运动的整体平动,到刚体的转动,我们所学习的物理也更加高深了。

      此外,在本次实验中,我也感受到了自己某些实验能力还有不足,例如对秒表的掌控等,希望在以后的实验课程中,能够提高自己的实验能力。

\end{document} 